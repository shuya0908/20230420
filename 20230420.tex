\documentclass{jsarticle}
% \documentclass[b4paper,landscape,14pt]{jsarticle}
\title{}
\author{}
\date{
% \number\year 年 \number\month 月
}
\usepackage{fenrir_v1_4_0}
\usepackage{ethm_v1_1_0}
\mathtoolsset{showonlyrefs=true}

\begin{document}
% \maketitle
\setcounter{section}{6}
\section{Brownian Motion and Partial Differential Equations}
\setcounter{subsection}{1}
\subsection{Brownian Motion and Harmonic Functions}

Prop. 7.7の(i)より境界条件 $g$ を満たすDirichlet問題の解が存在すれば,それは確率論的(probabilistic)な式 $u(x)=E_x[g(B_T)]$ で与えられることがわかる.
一方,$\partial D$ 上の(有界可測)関数 $g$ の選び方によらず,(ii)より確率論的な式は $D$ 上調和的な関数 $u$ を生み出すことがわかる.
たとえ $g$ に連続性が仮定されたとしても,関数 $u$ が境界条件 $g$ を満たすことは明らかではなく,また一般的に $g$ の連続性は不要(Exercise 7.24, 7.25参照,解をもたないDirichlet問題の例).

\begin{df*}
    $y\in\partial D$ とする.
    
    $D$ が $y$ における\textbf{外部錐条件 (exterior cone condition, ECC)}を満たす
    
    $\darrow$
    $\Exists \mathcal{C}(\neq\0)$:頂点 $y$ の開錐, $\Exists r>0$ s.t. $\mathcal{C}\cap B_r(y)\subset D^c.$
\end{df*}

\begin{ex*}
    凸領域は境界に含まれる任意の点において(ECC)を満たす.
\end{ex*}

\setcounter{thm}{8}
\begin{lem}~
    \begin{itemize}
        \item $D\subset \real^d$:有界領域
        \item $D$ が $\Forall y\in\partial D$ において(ECC)を満たす
        \item $T=\inf\set{t\ge0}{B_t\notin D}$
    \end{itemize}

    $\implies \Forall \eta>0$ に対し $\lim_{x\to y, x\in D}P_x(T>\eta) = 0.$
\end{lem}

\setcounter{thm}{7}
\begin{thm}[Dirichlet問題の解]
    直前のLemmaと同じ条件を仮定
    
    $\implies \Forall g$:連続 on $\partial D, u(x) = E_x[g(B_T)]$ は境界条件 $g$ をみたすDirichlet問題の唯一解を与える.
\end{thm}

\begin{proof}
    $\Forall y\in\partial D$:固定

    $g$:有界可測 on $\partial D$ なので,Prop. 7.7(ii)より $u$ は $D$ 上調和的.
    よって
    \setcounter{equation}{1}
    \begin{align}\label{eq:702}
        \lim_{x\to y, x\in D}u(x)
        = g(y)
    \end{align}
    が示せれば結論が言える.

    $\ep>0$ とする.
    $g$:有界連続 on $\partial D$ より
    \begin{itemize}
        \item 
        $\Forall \ep>0, \Exists \delta>0\text{ s.t. }\Forall z\in\partial D, \abs{z-y}<\delta\implies \abs{g(z)-g(y)}<\ep/3.$
        \item
        $\Exists M>0$ s.t. $\sup_{z\in\partial D}\abs{g(z)}\le M.$
    \end{itemize}

    $\implies \Forall \eta>0$ に対し
    \begin{align}
        \abs{u(x)-g(y)}
        &= \abs{E_x[g(B_T)-g(y)]} \\
        &\le E_x[\abs{g(B_T)-g(y)}\bm{1}_{\{T\le\eta\}}]
        + E_x[\abs{g(B_T)-g(y)}\bm{1}_{\{T>\eta\}}] \\
        &\le 
        \begin{multlined}[t]
            E_x[\abs{g(B_T)-g(y)}\bm{1}_{\{T\le\eta\}}\bm{1}_{\{\sup_{t\le\eta}\abs{B_t-x}\le\delta/2\}}] \\
            + E_x[\abs{g(B_T)-g(y)}\bm{1}_{\{\sup_{t\le\eta}\abs{B_t-x}>\delta/2\}}]
            + E_x[\abs{g(B_T)-g(y)}\bm{1}_{\{T>\eta\}}]
        \end{multlined} \\
        &\le 
        \begin{multlined}[t]
            E_x[\abs{g(B_T)-g(y)}\bm{1}_{\{T\le\eta\}\cap\{\sup_{t\le\eta}\abs{B_t-x}\le\delta/2\}}] \\
            + 2MP_x(\sup_{t\le\eta}\abs{B_t-x}>\delta/2)
            + 2MP_x(T>\eta)
            =: A_1+A_2+A_3.
        \end{multlined}
    \end{align}
\end{proof}

\end{document}
