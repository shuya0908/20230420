\documentclass{jsarticle}
% \documentclass[b4paper,landscape,14pt]{jsarticle}
\title{}
\author{}
\date{
% \number\year 年 \number\month 月
}
\usepackage{fenrir_v1_4_0}
\usepackage{ethm_v1_1_0}
\mathtoolsset{showonlyrefs=true}

\begin{document}
% \maketitle
\setcounter{section}{6}
\section{Brownian Motion and Partial Differential Equations}
\setcounter{subsection}{1}
\subsection{Brownian Motion and Harmonic Functions}

Prop. 7.7の(i)より境界条件 $g$ を満たすDirichlet問題の解が存在すれば,それは確率論的(probabilistic)な式 $u(x)=E_x[g(B_T)]$ で与えられることがわかる.
一方,$\partial D$ 上の(有界可測)関数 $g$ の選び方によらず,(ii)より確率論的な式は $D$ 上調和的な関数 $u$ を生み出すことがわかる.
たとえ $g$ に連続性が仮定されたとしても,関数 $u$ が境界条件 $g$ を満たすことは明らかではなく,また一般的に $g$ の連続性は不要(Exercise 7.24, 7.25参照,解をもたないDirichlet問題の例).

\begin{df*}
    $y\in\partial D$ とする.
    
    $D$ が $y$ における\textbf{外部円錐条件 (exterior cone condition, ECC)}を満たす
    
    $\darrow$
    $\Exists \mathcal{C}(\neq\0)$:頂点 $y$ の開円錐, $\Exists r>0$ s.t. $\mathcal{C}\cap B_r(y)\subset D^c.$
\end{df*}

\begin{ex*}
    凸領域は境界に含まれる任意の点において(ECC)を満たす.
\end{ex*}



\end{document}
