\documentclass{jsarticle}
% \documentclass[b4paper,landscape,14pt]{jsarticle}
\title{}
\author{}
\date{
% \number\year 年 \number\month 月
}
\usepackage{fenrir_v1_4_0}
\usepackage{ethm_v1_1_0}
\mathtoolsset{showonlyrefs=true}

\begin{document}
% \maketitle
\setcounter{section}{6}
\section{Brownian Motion and Partial Differential Equations}
\setcounter{subsection}{1}
\subsection{Brownian Motion and Harmonic Functions}

Prop. 7.7の(i)より境界条件 $g$ を満たすDirichlet問題の解が存在すれば,それは確率論的(probabilistic)な式 $u(x)=E_x[g(B_T)]$ で与えられることがわかる.
一方,$\partial D$ 上の(有界可測)関数 $g$ の選び方によらず,(ii)より確率論的な式は $D$ 上調和的な関数 $u$ を生み出すことがわかる.
たとえ $g$ に連続性が仮定されたとしても,関数 $u$ が境界条件 $g$ を満たすことは明らかではなく,また一般的に $g$ の連続性は不要(Exercise 7.24, 7.25参照,解をもたないDirichlet問題の例).


\begin{df*}
    $y\in\partial D$ とする.
    
    $D$ が $y$ における\textbf{外部錐条件 (exterior cone condition, ECC)}を満たす
    
    $\darrow$
    $\Exists \mathcal{C}(\neq\0)$:頂点 $y$ の開錐, $\Exists r>0$ s.t. $\mathcal{C}\cap \mathcal{B}_r(y)\subset D^c.$
\end{df*}

\begin{ex*}
    凸領域は境界に含まれる任意の点において(ECC)を満たす.
\end{ex*}


\setcounter{thm}{8}
\begin{lem}~
    \begin{itemize}
        \item $D\subset \real^d$:$\Forall y\in\partial D$ において(ECC)を満たす有界領域
        \item $T=\inf\set{t\ge0}{B_t\notin D}$
    \end{itemize}

    $\implies \Forall \eta>0$ に対し $\lim_{x\to y, x\in D}P_x(T>\eta) = 0.$
\end{lem}

\begin{proof}
    $\Forall u\in\real^d$ with $\abs{u}=1, \Forall \gamma\in(0, 1)$ に対し,原点頂点の開円錐
    \begin{align}
        \mathcal{C}(u, \gamma)
        := \set{z\in\real^d}{z\cdot u>(1-\gamma)\abs{z}}
    \end{align}
    を考える.
    与えられた $y\in\partial D$ に対し $r>0, u, \gamma$ を
    \begin{align}
        y+(\mathcal{C}(u, \gamma)\cap\mathcal{B}_{r})\subset D^c
    \end{align}
    を満たすように固定すると,$D$ は(ECC)を満たす($\mathcal{B}_{r}=\set{z\in\real^d}{\abs{z}<r}$).
    ここで
    \begin{itemize}
        \item $\mathcal{C}=\mathcal{C}(u, \gamma)\cap\mathcal{B}_{r}$
        \item $\mathcal{C}'=\mathcal{C}(u, \gamma/2)\cap\mathcal{B}_{r/2}$
    \end{itemize}
    と定める.
    
    \begin{screen}
        $\because)$
        $\mathcal{C}(u, \gamma/2)\subset \mathcal{C}(u, \gamma)$ となること:
        $1-\gamma/2>1-\gamma$ より
        \begin{align}
            z\in\mathcal{C}(u, \gamma/2)
            &\iff z\cdot u>(1-\gamma/2)\abs{z} \\
            &\implies z\cdot u>(1-\gamma)\abs{z} \\
            &\iff z\in\mathcal{C}(u, \gamma).
        \end{align}
    \end{screen}

    $\Forall F\subset \real^d$(開集合)に対し $T_F:=\set{t\ge0}{B_t\in F}$ と定める.
    このとき $\{T_{\mathcal{C}(u, \gamma/2)}=0\}=\bigcap_{s>0}\UB{\{T_{\mathcal{C}(u, \gamma/2)}<s\}}{\in\clf_s\text{ by Prop. 3.9(i)}}\in\clf_{0+}$ に注意すると,Blumenthalの0-1法則より
    \begin{align}
        P_0(T_{\mathcal{C}(u, \gamma/2)}=0)
        &= \lim_{s\downarrow0}P_0(T_{\mathcal{C}(u, \gamma/2)}<s) \\
        &\ge \lim_{s\downarrow0}P_0(B_s\in\mathcal{C}(u, \gamma/2)) > 0.
    \end{align}
    \begin{gather}
        \therefore 
        P_0(T_{\mathcal{C}(u, \gamma/2)}=0) = 1.
    \end{gather}

    これとサンプルパスの連続性より $P_0(T_{\mathcal{C}'}=0) = 1$ なので $T_{\mathcal{C}'}=0, P_0$-a.s.\nazo

    一方 $\Forall a\in(0, r/2)$ に対し
    \begin{align}
        \mathcal{C}'_a
        := \set{z\in\mathcal{C}'}{\abs{z}>a}
        = \mathcal{C}'\setminus \bar{\mathcal{B}_{a}}
        (\subset \mathcal{C}')
    \end{align}
    と定める.
    定め方より $\lim_{a\downarrow0}\uparrow\mathcal{C}'_a=\mathcal{C}'\implies \lim_{a\downarrow0}\downarrow T_{\mathcal{C}'_a}=T_{\mathcal{C}'}=0, P_0$-a.s.(対象の集合が大きくなると,より小さな時間で到達するイメージ)

    ゆえに 
    \begin{align}
        \Forall \eta>0, \Exists \alpha>0\text{ s.t. }
        & a<\alpha\implies P_0(T_{\mathcal{C}'_a}\le\eta)=1 \\
        \implies 
        \Forall \beta>0, \Forall \eta>0, \Exists \alpha>0\text{ s.t. }
        & a<\alpha\implies P_0(T_{\mathcal{C}'_a}\le\eta)\ge1-\beta.
    \end{align}

    この $a$ を固定する.
    ここで
    \begin{itemize}
        \item 
        $y+\mathcal{C}\subset D^c\implies T(=T_{D^c})\le T_{y+\mathcal{C}}$
        \item
        $\lim_{x\to y, x\in D}(y-x+\mathcal{C})\supset \mathcal{C}'_a\implies T_{y-x+\mathcal{C}}\le T_{\mathcal{C}'_a}$
    \end{itemize}
     より $\abs{y-x}$ を十分小さくとると
     \begin{align}
         P_x(T\le\eta)
         &\ge P_x(T_{y+\mathcal{C}}\le\eta) \\
         &= P_0(T_{y-x+\mathcal{C}}\le\eta) \\
         &\ge P_0(T_{\mathcal{C}'_a}\le\eta)
         \ge 1-\beta.
     \end{align}
     $$
     \therefore 
     \lim_{x\to y, x\in D}P_x(T<\eta)<\beta.
     $$

     $\beta$ の任意性より $\lim_{x\to y, x\in D}P_x(T<\eta)=0.$
\end{proof}


\setcounter{thm}{7}
\begin{thm}[Dirichlet問題の解]
    直前のLemmaと同じ条件を仮定
    
    $\implies \Forall g$:連続 on $\partial D, u(x) = E_x[g(B_T)]$ は境界条件 $g$ をみたすDirichlet問題の唯一解を与える.
\end{thm}

\begin{proof}
    $\Forall y\in\partial D$:固定

    $g$:有界可測 on $\partial D$ なので,Prop. 7.7(ii)より $u$ は $D$ 上調和的.
    よって
    \setcounter{equation}{1}
    \begin{align}\label{eq:702}
        \lim_{x\to y, x\in D}u(x)
        = g(y)
    \end{align}
    が示せれば結論が言える.

    $\ep>0$ とする.
    $g$:有界連続 on $\partial D$ より
    \begin{itemize}
        \item 
        $\Forall \ep>0, \Exists \delta>0\text{ s.t. }\Forall z\in\partial D, \abs{z-y}<\delta\implies \abs{g(z)-g(y)}<\ep/3.$
        \item
        $\Exists M>0$ s.t. $\sup_{z\in\partial D}\abs{g(z)}\le M.$
    \end{itemize}

    $\implies \Forall \eta>0$ に対し
    \begin{align}
        \abs{u(x)-g(y)}
        &= \abs{E_x[g(B_T)-g(y)]} \\
        &\le E_x[\abs{g(B_T)-g(y)}\bm{1}_{\{T\le\eta\}}]
        + E_x[\abs{g(B_T)-g(y)}\bm{1}_{\{T>\eta\}}] \\
        &\le 
        \begin{multlined}[t]
            E_x[\abs{g(B_T)-g(y)}\bm{1}_{\{T\le\eta\}}\bm{1}_{\{\sup_{t\le\eta}\abs{B_t-x}\le\delta/2\}}] \\
            + E_x[\abs{g(B_T)-g(y)}\bm{1}_{\{\sup_{t\le\eta}\abs{B_t-x}>\delta/2\}}]
            + E_x[\abs{g(B_T)-g(y)}\bm{1}_{\{T>\eta\}}]
        \end{multlined} \\
        &\le 
        \begin{multlined}[t]
            E_x[\abs{g(B_T)-g(y)}\bm{1}_{\{T\le\eta\}\cap\{\sup_{t\le\eta}\abs{B_t-x}\le\delta/2\}}] \\
            + 2MP_x(\sup_{t\le\eta}\abs{B_t-x}>\delta/2)
            + 2MP_x(T>\eta)
            =: A_1+A_2+A_3.
        \end{multlined}
    \end{align}
\end{proof}

\end{document}
